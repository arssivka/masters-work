\section*{Выводы по главе 1}
\addcontentsline{toc}{section}{Выводы по главе 1}

При разработке вышеперечисленных платформ, авторы делают упор на удобство разработки и абстракции, специфичные для области разработок робототехнических систем. Также, в каждом из них основным свойством является модульность. Многие фреймворки используют алгоритмы с большой вычислительной сложностью, из-за чего не пригодны для систем с требованием отклика реального времени. Часть описанных проблем решаются в ROS 2.0, цель которого предоставить высокопроизводительный кластер с упором на высокоскоростное сетевое взаимодействие.

Данный анализ показал актуальность проблемы с выбором инструмента для разработки программных фреймворков для мобильных робототехнических систем и дает представление о сильных и слабых сторонах различных решений при дальнейшем проектировании. 