\section*{Выводы по главе 2}
\addcontentsline{toc}{section}{Выводы по главе 2}

В данной главе представленны результаты исследования современных алгоритмов 
сериализации и межпоточной синхронизации и были представленны два прототипа 
архитектуры робототехнической модульной системы с использованием 
высокопроизводительных очередей без блокировок. При проектировании прототипов 
помимо достижения высокого времени отклика системы учитывалась возможность в 
дальнейшем использовать языки программирования, отличных от C++, и 
использование библиотеки за пределами Unix-подобных операционных систем.

В области современных высоконагруженных систем активно исследуются и разрабатываются различные алгоритмы и примитивы синхронизации без блокировок для повышения производительности.  Особенность разработанных прототипов заключается в том, что они не используют примитивы синхронизации с блокировками. Для достижения поставленной задачи был спроектирован механизм межмодульной коммуникации с отложенным исполнением без системных вызовов, в котором компоненты не могут напрямую взаимодействовать друг с другом. Для взаимодействия модулям предоставлен API, который позволяет передавать данные и вызывать зарегистрированные в системе функции из других модулей.

Для возможности дальнейшего расширения системы вся межмодульная коммуникация работает путем передачи указателей на дессериализированный буфер памяти. Данный подход позволяет существенно увеличить производительность при передачи больших блоков памяти, например, изображения при условии, что десериализация происходит без копирования. Синхронный прототип использует структуры из Google Protobuf с использованием таблицы для определения типа. Данное решение позволяет увеличить скорость за счет отсутствия затрат на десериализацию, но при этом система должна иметь фиксированный набор возможных сообщений. В асинхронной версии данная проблема решенна с использованием библиотеки Google Flat Buffers. При этом архитектура позволяет исполюзовать любой другой сериализатор.