\section{ROS}

ROS (Robot Operation System) - Операционная система для роботов - это один из самых известных на сегодняшний день инструментов для разработки программного обеспечения для роботов. Данный фреймворк был разработан в 2007 году в лаборатории искусственного интеллекта Стэндфордского университета и распространяется под открытой BSD лицензией. Развитие проекта контролируется некоммерческой организацией Open Source Robotics Foundation.

ROS не является операционной системой в привычном смысле. Данный фреймворк является многопоточным гетерогенным вычислительным кластером для Linux-подобных систем, который предоставляет разработчику набор готовых решений, в том числе для взаимодействия с аппаратным уровнем и обмена сообщениями между процессами. Фреймворк активно развивается с участием множества крупных нучных центров и компаний и имеет большой набор готовых решений в области распознавания объектов, сегментации, распознавании лиц, стереозрения, планирования движений и маршрутов и многих других областях. Помимо языка C++, на котором реализованно ядро системы, ROS имеет подержку множества других языков: Java, Python, Lisp, MATLAB.

В основе системы лежит ядро roscore, которое предоставляет основные механизмы для взаимодействия между модулями системы, запущенных в отдельных процессах, через сокеты Беркли и работает независимо от запущенных модулей. Модули системы, которые именуются нодами (англ. node),  являются самостоятельными программами, которые запускаются в отдельном процессе и общаются с другими помпонентами через ядро с использованием протокола удаленных вызовов xmlrpc. Для уменьшения нагрузки на процессор каждый нод использует определенную частоту основного цикла обработки событий. Многопоточная безопасность исключает гонку за данные из-за того, что каждый модуль запускается в отдельном потоке и имеет свое собственное изолированное адресное пространство.

Данное решение позволяет легко расширять и переиспользовать различные программные компоненты, хорошо использует ресурсы многоядерных систем, в том числе позволяет взаимодействовать с отдельными компонентами через сетевой протокол, но имеет ряд существенных недостатков. Как было ранее сказанно, ROS использует ресурсоемкие алгоритмы для межмодульного общения: блокирующая синхронизация, избыточность протокола xmlrpc по сравнению с другими протоколами удаленных вызовов, множественное копирование при передаче сообщений. Перечисленные недостатки могут существенно снизить общую производительность системы на маломощных робототехнических системах. Отдельно можно выделить, что из-за большой экосистемы различных модулей для обратной совместимости во фреймворке используются устаревшие версии языка C++ стандарта 2003 года и Python версии 2.7, из-за чего могут возникнуть трудности с использованием современных библиотек. Данная проблема решена в ROS Kinetic Kame, но большинство модулей могут иметь проблемы с совместимостью.


