\section{ROS 2.0}

Первая версия ROS изначально проектировалась для робота PR2, разработанного калифорнийской робототехнической лабораторией Willow Garage. Использование фреймворка предполагало, что робот имеет высокие вычислительные ресуры, хорошее беспроводное соединение и не требуется время отклика в реальном времени. Для преодоления этих ограничений под руководством Open Source Robotics Foundation началась разработка ROS 2.0. На данный момент проект находится на стадии разработки. Ориентировочная дата релиза запланированна на декабрь 2017 года.

По информации, представленной на официальном портале фреймворка, в основе ядра 
новой системы будут лежать такие технологии, как: Zeroconf, Google Protocol 
Buffers, ZeroMQ (and the other MQs), Redis, WebSockets, DDS (Data Distribution 
Service). Из этого следует, что система расчитанна преимущественно на 
высокоскоростное сетевое взаимодействие в группе между различными РТК. Так же 
разработчики пытаются сохранить совместимость с существующей программной базой, 
предоставляемой ROS первой версии.