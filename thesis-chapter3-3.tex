\section{Тестирование производительности}

Для замера производительности использовалась библиотека Celero.

Тестовый сценарий был имплементирован для того, чтобы протестировать производительность версии системы, реализованной с использованием google protocol buffers. В этом тестовом сценарии создавалось определенное относительно большое количество модулей. Эти модули подписаны на получение определенного сообщения. Во время выполнения этого сценария, определенное относительно большое количество сообщений отправляется в эти модули. В таблице \ref{tab:sync_bench} представлены результаты тестов.

\begin{table}
    \caption{\label{tab:sync_bench}Результаты тестов}
    \begin{center}
        \begin{tabularx}{\textwidth}{|c|X|X|X|X|}
            \hline
            & \multicolumn{4}{|c|}{Количество сообщений} \\
            \hline
            Кол-во модулей & 100   & 1000   & 10000   & 100000   \\
            \hline
            10             & 0 мс  & 3 мс   & 37 мс   & 346 мс   \\
            \hline
            100            & 2 мс  & 15 мс  & 158 мс  & 1568 мс  \\
            \hline
            1000           & 18 мс & 129 мс & 1251 мс & 13508 мс \\
            \hline
        \end{tabularx}
    \end{center}
\end{table}

На данный момент, средняя частота программного взаимодействия между операционной системой и программным обеспечением мобильного работа находится на отметке примерно 14 миллисекунд. Это частота была получена на популярном мобильном роботе Darwoin OP[ссылку на дарвина вставь, да]. Наиболее вероятно, количество модулей и передаваемых сообщений в практическом использовании подобных робототехнических систем будет схоже с тем количеством, которое представлено в таблице 1 в первом тесте, то есть порядка 10 модулей и 100 сообщений. Предложенной робототехническая система справляется с такой нагрузкой меньше чем за 1 миллисекунду. Это значительно быстрее, чем скорость программного взаимодействия с аппаратным обеспечением робота. Самый худший случай нагрузки, представленный в тестах - это 1000 модулей и 100000 сообщений. По представленным результатам видно, что платформа справляется с такой нагрузкой за 13 секунд. В ситуациях реального времени, это неприемлемая скорость. Однако, сложно представить себе мобильного автономного робота, которому необходимо такое количество модулей. В наши дни, не существует такой автономной робототехнической системы. В доказательство выводов по производительности системы в главе Что-то про реальное применение платформы представлен реальный пример использования робототехнической платформы в практической задаче.