\section*{Выводы по главе 3}
\addcontentsline{toc}{section}{Выводы по главе 3}

Тестирование библиотеки велось одновременно с ее разработкой. Многопоточные приложения создают большое количество трудновоспроизводимых ситуаций, когда может нарушиться работа всей системы. Автоматизация тестирования через систему непрерывной интеграции с использованием юнит-тестирования и автоматического поиска возможных утечек памяти позволила своевременно устранить большое количество ошибок.

Результаты производительности разработанных библиотек показывают средний прирост производительности в 5-8 раз в сравнении с аналогами, использующие блокирующую синхронизаицю. Существенный прирост в производительности наблюдается при увеличении количества потоков на многоядерной платформе при частом обращении нескольких потоков к одному участку кода. Среднее время передачи 100 млн. сообщений составляет 1-2 секунды с использованием core i5. Полученные результаты показывают, что разработанные алгоритмы синхронизации предоставляют интерес не только на мобильных роботах, но и на высокопроизводительных рабочих станциях.

Средние затраты времени на межмодульную синхронизацию при 10 модулях и 100 сообщений за одну итерацию составляют десятые доли миллисекунды. Данная конфигурация системы наиболее вероятна для небольшого мобильного робота, например, для участия в робототехническом чемпионате Robo Cup. При этом среднее время взаимодействия программного обеспечения с аппаратной частью у мобильного гуманоидного робота Darwin OP составляет 14 миллисекунд.  Из этого можно сказать, что разработанные алгоритмы могут эффективно применяться на мобильных аппаратных платформах.