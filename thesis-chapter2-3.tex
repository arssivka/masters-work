\section{Проектирование асинхронного фреймворка}

При проектировании асинхронной модели фреймворка была существенно изменена архитектура синхронного ядра. Так же при проектировании были учтены выявленные архитектурные ошибки. Данная реализация системы подразумевает использование высокоскоростных алгоритмов десериализации. При тестировании данной системы исползовалась библиотека Google Flat Buffers, но данная реализация позволяет применять любой алгоритм сериализации/десериализации с использованием бинарного буфера памяти.

Отличие асинхронной версии от синхронной в первую очередь заключается в том, что система сама уведомляет о всех изменениях: модули регистрируют функции обратного вызова для получения требуемой информации о состоянии системы при ее изменениях.

\textit{abstract\_launcher}

\textit{task\_queue}

\textit{lockfree\_task\_queue}

\textit{blocking\_task\_queue}

\textit{mechanism}

\textit{task\_scheduler}

\textit{core\_base}

\textit{topic\_mechanism}

\textit{service\_mechanism}

\textit{core}

