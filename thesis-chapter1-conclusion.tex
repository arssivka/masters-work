\section*{Выводы по главе 1}
\addcontentsline{toc}{section}{Выводы по главе 1}

При разработке вышеперечисленных платформ, авторы делают упор на 
удобство разработки и абстракции, специфичные для области 
разработок робототехнических систем. Также, в каждом из них 
основным свойством является модульность. Многие фреймворки 
используют алгоритмы с большой вычислительной сложностью, из-за 
чего не пригодны для систем с требованием отклика реального 
времени. Часть описанных проблем решаются в ROS 2.0, цель 
которого предоставить высокопроизводительный кластер с упором на 
высокоскоростное сетевое взаимодействие. В таблице 
\ref{tab:frameworks} приведены ключевые особенности описанных 
ранее фреймворков при использовании на мобильных РТК.

\begin{table}
    \caption{\label{tab:frameworks}Обзор фреймворков}
    \begin{center}
        \begin{tabularx}{\textwidth}{|c|X|X|X|}
            \hline
            Платформа & Особенности & Ограничения & Недостатки \\
            \hline
            ROS & Множество полезных инструментов от поддержки симуляторов различных визуализаторов & Эффективен только для UNIX-подобных системах с высокопроизводительной аппаратной поддержкой & Многопроцессная архитектура, множество зависимостей от сторонних библиотек \\
            \hline
            ROS 2.0 & Совместимость с ROS, поддержка 
            мультиагентных систем, эффективное сетевое 
            взаимодействие & Сетевое межмодульное взаимодействие 
            & Множество зависимостей от сторонних библиотек, на 
            стадии разработки \\
            \hline
            OROCOS & Время отклика в реальном времени, 
            оптимизировано использование памяти & Разработан 
            только для исследовательских целей & Использование 
            скриптового языка Lua, что сказывается на 
            производительности \\
            \hline
            OPRoS & Утилита для визуального программирования & Разработан только для исследовательских целей & Низкая производительность \\
            \hline
            Urbi & Использует свой язык программирования ориентированный для использования в робототехнике, кроссплатформенность & Использование собственного языка программирования с использованием нестандартной парадигмы & Нетрадиционный способ реализации многопоточности, собственный узкозаточенный язык программирования \\
            \hline
            PX4 & Управление в стиле unix-shell, возможность взаимодействия с ROS & Предназначен для использования на микроконтроллерах & Основан с использованием стандарта POSIX \\
            \hline
        \end{tabularx}
    \end{center}
\end{table}

Данный анализ показал актуальность проблемы с выбором инструмента для разработки программных фреймворков для мобильных робототехнических систем и дает представление о сильных и слабых сторонах различных решений при дальнейшем проектировании. 