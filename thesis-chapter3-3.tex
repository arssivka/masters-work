\section{Тестирование производительности}

В данном разделе приводятся результаты тестирования производительности межмодульной коммуникации разработанных библиотек. Так же приводится сравнение с аналогоми из библиотеки boost. Тестирование производится на двух аппаратных платформах на базе следующих процессоров: Intel Atom z530 1.6ГГц и Intel Core i5-6300HQ 2.30GГГц. Данные результаты позволяют сравнить производительность как на мобильных системах, так и при использовании библиотеки на рабочих станциях, нарпимер, в качестве удаленного клиента. Тестирование происходит на операционной системе Arch Linux с ядром версии 4.10.13. Для замера производительности используется библиотека Celero, которая использует высокочастотные таймеры (1 МГц) для подсчета затраченного времени.

Далее приводятся результаты тестирования для системы рассылки сообщений. Поскольку система удаленных вызовов процедур работает аналогичным образом, то результаты производительности приблизительно равны.

Все тестовые сценарии работают по следующему алгоритму:

\begin{enumerate}
    \item Добавить \textit{N} слушателей/модулей в систему
    \item Синхронизировать операции
    \item Отправить \textit{M} сообщений
    \item Синхронизировать операции
\end{enumerate}

В каждом тесте суммарно передается \textit{M * N} сообщений. Данные тесты позволяют проверить влияние на производительность как количества сообщений, так и количества модулей. Тесты рассылки сообщений исполняются в одном потоке т.к. в данном случае демонстрируется наибольшая потеря производительности. Так же в тестах с использованием библиотеки Boost отсутствует операция поиска нужного топика и слушатели добавляются сразу напрямую в требуемый объект, что существенно ускоряет добавление слушателей.

Сравнение производительности для Boost Asio производится только для пула задач, т.к. архитектура механизмов при использовании \textit{boost::asio::io\_service} в качестве лаунчера система мезанихмов остается неизменной.


\begin{table}[htb]
    \caption{\label{tab:core_rrc}Синхронная рассылка сообщений (core i5)}
    \begin{center}
        \begin{tabularx}{\textwidth}{|c|X|X|X|X|}
            \hline
            & \multicolumn{4}{|c|}{Количество сообщений} \\
            \hline
            Кол-во модулей & 100   & 1000   & 10000   & 100000   \\
            \hline
            10             & 0.059 мс  & 0.456 мс   & 4.621 мс   & 49.460 мс   \\
            \hline
            100            & 0.28 мс  & 1.782 мс  & 21.069 мс  & 221.779 мс  \\
            \hline
            1000           & 3.296 мс & 27.85 мс & 275.936 мс & 2713.869 мс \\
            \hline
        \end{tabularx}
    \end{center}
\end{table}

\begin{table}[htb]
    \caption{\label{tab:core_oldrrc}Асинхронная рассылка сообщений (core i5)}
    \begin{center}
        \begin{tabularx}{\textwidth}{|c|X|X|X|X|}
            \hline
            & \multicolumn{4}{|c|}{Количество сообщений} \\
            \hline
            Кол-во модулей & 100   & 1000   & 10000   & 100000   \\
            \hline
            10             & 0.047 мс  & 0.437 мс   & 4.176 мс   & 45.404 мс   \\
            \hline
            100            & 0.223 мс  & 1.644 мс  & 16.682 мс  & 170.504 мс  \\
            \hline
            1000           & 1.842 мс & 14.176 мс & 146.31 мс & 1461.084 мс \\
            \hline
        \end{tabularx}
    \end{center}
\end{table}

\begin{table}[htb]
    \caption{\label{tab:core_signals2}Рассылка сообщений через Boost Signal2 (core i5)}
    \begin{center}
        \begin{tabularx}{\textwidth}{|c|X|X|X|X|}
            \hline
            & \multicolumn{4}{|c|}{Количество сообщений} \\
            \hline
            Кол-во модулей & 100   & 1000   & 10000   & 100000   \\
            \hline
            10             & 0.094 мс  & 0.777 мс   & 7.779 мс   & 801.6 мс   \\
            \hline
            100            & 0.578 мс  & 5.288 мс  & 56.391 мс  & 547.934 мс  \\
            \hline
            1000           & 5.964 мс & 56.390 мс & 570.35 мс & 5595.921 мс \\
            \hline
        \end{tabularx}
    \end{center}
\end{table}


На данный момент, средняя частота программного взаимодействия между операционной системой и программным обеспечением мобильного работа находится на отметке примерно 14 миллисекунд. Это частота была получена на популярном мобильном роботе Darwoin OP \cite{ha2011development}. Наиболее вероятно, количество модулей и передаваемых сообщений в практическом использовании подобных робототехнических систем будет схоже с тем количеством, которое представлено в таблице 1 в первом тесте, то есть порядка 10 модулей и 100 сообщений за одну итерацию. Предложенная робототехническая система справляется с такой нагрузкой меньше чем за сотые дли миллисекунды. Это значительно быстрее, чем скорость программного взаимодействия с аппаратным обеспечением робота. Самый худший случай нагрузки, представленный в тестах - это 1000 модулей и 100000 сообщений. По представленным результатам видно, что платформа справляется с такой нагрузкой за 1-2 секунды, взависимости от реализации. В ситуациях реального времени, это неприемлемая скорость. Однако, сложно представить себе мобильного автономного робота, которому необходимо такое количество модулей и обрабатывающий такое количество данных. Как уже было сказанно ранее, в наши дни, не существует такой автономной робототехнической системы.

В таблице \ref{tab:core_asio} приведены результаты тестов сравнения производительности очереди задач в RRC и Boost Asio. 

\begin{table}[htb]
    \caption{\label{tab:core_asio}Скорость добавления задач в очередь (core i5)}
    \begin{center}
        \begin{tabularx}{\linewidth}{|c|X|X|}
            \hline
            К-во потоков & RRC & Boos Asio  \\
            \hline
            1 & 231 мкс & 712 мкс \\
            \hline
            2 & 167 мкс & 1468 мкс \\
            \hline
            4 & 136 мкс & 1815 мкс \\
            \hline
            8 & 113 мкс & 1952 мкс \\
            \hline
            16 & 55 мкс & 2016 мкс \\
            \hline
            32 & 29 мкс & 2119 мкс \\
            \hline
        \end{tabularx}
    \end{center}
\end{table}

По результатам видно, что при увеличении потоков при обращении к одереди без блокировок в разработанной системе возрастает общее время отклика. Boost Asio активно использует системные механизмы и критические секции из-за чего при частом возникновении ситуации гонки общая производительность системы падает.