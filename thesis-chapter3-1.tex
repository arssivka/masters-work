\section{Юнит-тестирование}

Для поиска возможных ошибок в реализованных библиотеках был использован фреймворк для юнит-тестирования (unit-tesing) Google Test. Все реализации фреймворков разбиты на большое количество служебных классов для возможности более эффективного покрытия тестами каждого компонента фреймворка. В диаграммах классов большинство данных объектов не было отраженно, т.к. выполняют небольшое количество подзадач и используются исключительно внутри системы без прямого доступа к ним. В ходе тестирование было выявленно достаточно большое количество ошибок, приводящих к неправильной логике работы и падениям системы.

Репозиторий исходного кода проекта был подключен к системе непрерывной интеграции Travis CI с автоматической проверкой всех внесенных изменений в исходный код. Система автоматически компилирует исходный код в версях с отладочной информацией и без нее, запускает для них исполняемый файл для юнит-тестирования и публикует результат сборки для каждого коммита в репозиторий. Данный инструмент позволил своевременно обнаружить ошибки при слиянии различных изменений в репозиторий и сохранить целостность программного кода.