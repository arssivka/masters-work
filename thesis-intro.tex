\chapter*{ВВЕДЕНИЕ}
\addcontentsline{toc}{chapter}{ВВЕДЕНИЕ}

На сегодняшний день робототехника является довольно молодой и активно 
развивающейся прикладной междисциплинарной наукой. Современные 
робототехнические комплексы (РТК) отличаются широким набором встроенных 
программно-аппаратных средств, реализующих анализ окружающей ситуации, связь, 
управление исполнительными механизмами и другие специализированные функции. 
Одной из основных проблем в данной области можно считать избыточные трудозатраты при проектировании программных компонентов для управления робототехнической системой. Особенно остро она выражается на мобильных роботах, в которых для повышения автономности используется энергоэффективные аппаратные вычислительные устройства с низкой производительностью. Так же в большинстве случаев разработчикам программного обеспечения для РТК приходиться интегрировать различные гетерогенные программно-аппаратные компоненты, вследствие чего возникает ряд проблем, связанных с унификацией протоколов коммуникации и одновременным управлением различными распределенными средствами, а так же с переносимостью и повторного использования реализованного программного кода между различными робототехническими системами.

Для решения проблем с программно-аппаратной коммуникацией и унификации 
программных модулей в робототехнике активно развиваются различные 
робототехнические фреймворки, которые решают большой класс задач для упрощения разработки робототехнических систем. Большинство современных фреймворков, такие как Robot Operation System (ROS), хорошо решают задачи проектирования сложных робототехнических систем, их отладки и переиспользования программного кода, но требует высоких вычислительных затрат для межмодульной коммуникации, из-за чего существенно снижает производительность всей системы на слабых мобильных аппаратных платформах, таких как Darwin-OP на базе intel atom z530. Роботы данного класса активно используются для исследований и в различных соревнованиях с участием роботов, где часто требуется быстрая реакция в зависимости от изменений в окружающей среде. Среди недостатков большинства популярных фреймворков можно так же выделить отсутствие поддержки операционных систем, отличных от unix-подобных, и современных стандартов языка программирования C++ начиная с ISO/IEC 14882:2011 \cite{c++2011iso} и выше.

Цель данной работы заключается в исследовании возможных решений описанных ранее проблем и в разработке модульного асинхронного потокобезопасного робототехнического фреймворка с открытым исходным кодом. Фреймфорк ориентирован преимущественно на мобильную робототехнику с низкой вычислительной мощностью. Повышение производительности программной системы в настоящей работе предлагается обеспечивать за счет снижения использования количества блокирующих примитивов синхронизации в многопоточной среде путем использования неблокирующих (lock-free) аналогов и уменьшения количества операций копирования при межмодульном взаимодействии и десериализации сообщений (zero-copy deserialization).
