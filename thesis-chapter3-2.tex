\section{Тестирование на отказоустойчивость}

В разработанной системе активно используется рассылка сообщений через передачу 
указателя на ранее аллоцированный блок памяти, что требует контроля времени 
жизни данных буферов во избежании утечек памяти или исключить обращение к 
памяти ранее удаленного объекта. Так же в системе могут быть ошибки из-за 
возникновения ситуации гонок за данные.

Поиск возможных проблем данного рода производилось с использованием утилиты отладки использования памяти, обнаружения учетек памяти и провилирования Valgrind \cite{nethercote2007valgrind}. Поскольку данное тестирование производилось после реализации юнит-тестов, во всех реализациях фреймворка была обнаружена ошибка в механизме асинхронной версии фреймворка, которая приводила к вызову неинициализированного функтора. Данное тестирование было так же добавленно в скрипт автоматической сборки и запуска тестов непрерывной интеграции.

Архитектура библиотеки не предусматривает отказоустойчивость при возникновении исключений в пользовательских классах, поэтому обработка пользовательских ошибок возложена на пользователей библиотеки.